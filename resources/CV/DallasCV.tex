%!TEX TS-program = xelatex
\documentclass[]{CV}
\setcounter{secnumdepth}{-1}
\usepackage{multicol}
\usepackage{fontawesome}
\usepackage{hyperref}
\usepackage{fontspec}


%\renewcommand{\familydefault}{\sfdefault}

\begin{document}

\header{}{Tad Dallas}{}

\begin{aside}
\vspace{-2cm}
\includegraphics[width=\textwidth]{dallasLab2.pdf}
  \section{about}
    ~
		Assistant professor
		University of South Carolina
		Dept of Biological Sciences
    ~
    \faEnvelope \ \  \href{mailto:tad.a.dallas@gmail.com}{tad.a.dallas@gmail.com}
    \faDesktop  \ \ \href{https://taddallas.github.io}{taddallas.github.io}
    \faGithub   \ \ \ \href{http://github.com/taddallas}{taddallas}
    ~
  \section{programming}
   ~
   {\mefont Proficient}
   R
   Matlab/Octave
   SQL
   ~
   {\mefont Familiar }
   C++
   julia
   Python
   ~
   {\mefont Markup}
   \LaTeX
   Markdown
   HTML/XML/XPath
   ~
   {\mefont Version control}
   git
\end{aside}





\section{\faFlask \ \ experience}

\begin{entrylist}

  \entry
    {2022 - }
    {Assistant professor}
    {University of South Carolina, \textit{Columbia, SC}}
    {\emph{Dept. of Biological Sciences}} \\
    
  \entry
    {2019 - 2021}
    {Assistant professor}
    {Louisiana State University, \textit{Baton Rouge, LA}}
    {\emph{Dept. of Biological Sciences}} \\


  \entry
    {2019}
    {Visiting researcher}
    {International University of Rijeka, \textit{Croatia}}
    {\emph{Dept. of Mathematics}} \\

  \entry
    {2019}
    {Visiting researcher}
    {CSIC, \textit{Estación Biológica de Do\~nana}, Spain}
    {\emph{Lab of Pedro Jordano}} \\

  \entry
    {2018 - 2019}
    {Postdoctoral fellow}
    {University of Helsinki - \textit{Centre for Ecological Change}}
    {\emph{Advised by Otso Ovaskainen}} \\


  \entry
    {2016 - 2018}
		{Postdoctoral fellow}
    {University of California--Davis - \textit{Center for Population Biology}}
    {\emph{Advised by Alan Hastings}} \\


  \entry
    {2015}
    {Distributed $R$ Analytics Intern}
    {HP Vertica - Big Data Platform Dev Team}
    {\emph{Software development for analysis of large data}} \\

  \entry
    {2010-2011 \ \ \ \ \ }
    {Biological Science Technician}
    {USDA - Agricultural Research Service}
    {\emph{Subtropical Plant Pathology Lab }} \\

  \entry
    {2008}
    {Mathematical Biology Program}
    {NSF Research Experience for Undergraduates (REU)}
    {\emph{Mathematical estimation of host range using mark-recapture data}}
\end{entrylist}







\section{\faGraduationCap \  education}

\begin{entrylist}
  \entry
    {2011 - 2016}
    {\normalfont Ph.D. Ecology}
    {U Georgia - Odum School of Ecology}
    {\emph{Advised by John Drake}}\\

  \entry
    {2009 - 2010}
    {M.S. Biology}
    {Truman State University}
    {\emph{Ecology of small mammal-tick interactions} \\ advised by Stephanie For\'e}\\

 \entry
    {2005 - 2009 \ \ \ \ \ }
    {B.S. Biology}
    {Truman State University}
    {Majoring in Biology\\
    \emph{Minor in Mathematical Biology}}\\
\end{entrylist}














\section{\faBook \ \  publications}


{\yearfont pre-prints} 

\begin{itemize}

\item Richards, RL, Foster, G, Elderd, BD, \& {\mefont TA Dallas}. 2022. Comparing Waves of COVID-19 in the US: Scale of response changes over time. medRxiv. doi: 10.1101/2022.03.01.22271713 

\item {\mefont Dallas, TA}, SJ Ryan, B Bellekom, A Fagre,  R Christofferson, \& C Carlson. 2021. Predicting the tripartite network of mosquito-borne disease. ecoEvoRxiv. doi: 10.32942/osf.io/xzmp8

\end{itemize}




{\yearfont 2022}

\begin{itemize}

\item {\mefont TA Dallas}, G Foster, RL Richards \& BD Elderd. 2022. Epidemic time series similarity is related to geographic distance and age structure. \textit{Infectious Disease Modeling}. doi: 10.1016/j.idm.2022.09.002

\item {\mefont TA Dallas} \& D Kramer. 2022. A latitudinal signal in the relationship between species geographic range size and climatic niche area. \textit{Ecography}. doi: 10.1111/ecog.06349

\item G Foster, BD Elderd, RL Richards \& {\mefont TA Dallas}. 2022. Estimating $R_0$ from early exponential growth: Parallels between 1918 influenza and 2020 SARS-CoV-2 pandemics. \textit{PNAS Nexus}. doi: 10.1093/pnasnexus/pgac194

\item {\mefont TA Dallas}, C Carlson, P Stephens, SJ Ryan, D Onstad. 2022. \texttt{insectDisease}: programmatic access to the Ecological Database of the World's Insect Pathogens. \textit{Ecography} doi: 10.1111/ecog.06152

\item Ten Caten, C, LA Holian, \& {\mefont TA Dallas}. 2022. Effects of occupancy estimation on abundance-occupancy relationships. \textit{Biology Letters}. doi: 10.1098/rsbl.2022.0137

\item Antão,LH, B Weigel, G Strona, M Hällfors, E Kaarlejärvi, {\mefont TA Dallas}, \textit{et al.} 2022. Climate change reshuffles northern species within their niches. \textit{Nature Climate Change}. doi: 0.1038/s41558-022-01381-x

\item Carlson, CJ, \textit{et al.} 2022. The Global Virome in One Network (VIRION): an Atlas of Vertebrate-Virus Associations. \textit{mBio}. doi: 10.1128/mbio.02985-21

\item Fuzessy, L, \textit{et al.} 2022. Functional roles of frugivores and plants shape hyper-diverse mutualistic interactions under two antagonistic conservation scenarios. \textit{Biotropica}. doi: 10.1111/btp.13065

\item Smolander, OP, \textit{et al.} 2022. Improved chromosome-level genome assembly of the Glanville fritillary butterfly (\textit{Melitaea cinxia}) integrating Pacific Biosciences long reads and a high-density linkage map. \textit{GigaScience} doi: 10.1093/gigascience/giab097

\item {\mefont Dallas, TA} \& P Jordano. 2022. Parasite species richness and host range are not spatially conserved. \textit{Global Ecology and Biogeography}. doi: 10.1111/geb.13452

\item Ten Caten, C, LA Holian, \& {\mefont TA Dallas}. 2022. Weak but consistent abundance-occupancy relationships across taxa, space, and time. \textit{Global Ecology and Biogeography}. doi: 10.1111/geb.13472

\item Becker,D, GF Albery, AR Sjodin, T Poisot, {\mefont TA Dallas}, EA Eskew, MJ Farrell, S Guth, BA Han, NB Simmons, CJ Carlson. 2022. Optimising predictive models to prioritise viral discovery in zoonotic reservoirs. \textit{Lancet Microbe} doi: 10.1016/S2666-5247(21)00245-7

\item {\mefont Dallas, TA} \& D Kramer. 2022. Temporal variability in population and community dynamics. \textit{Ecology}. doi: 10.1002/ecy.3577


\end{itemize}






{\yearfont 2021}

\begin{itemize}

\item Albery, GF. et al. 2021. The science of the host-virus network. \textit{Nature Microbiology} doi: 10.1038/s41564-021-00999-5

\item {\mefont Dallas, TA} \& P Jordano. 2021. Spatial variation in species roles in host-helminth networks. \textit{Philosophical Transactions B} doi: 10.1098/rstb.2020.0361

\item Farrell, MJ, AW Park, C Cressler, {\mefont TA Dallas}, S Huang, N Mideo, I Morales-Castilla, TJ Davies \& P Stephens. 2021. The ghost of hosts past: impacts of host extinction on parasite specificity. \textit{Philosophical Transactions B} doi: 10.1098/rstb.2020.0351

\item Morales-Castilla, I, P Pappalardo, MJ Farrell, AA Aguirre, S Huang, ALM Gehman, {\mefont TA Dallas}, D Gravel \& TJ Davies. 2021. Forecasting parasite sharing under climate change. \textit{Philosophical Transactions B} doi: 10.1098/rstb.2020.0360

\item Carlson, CJ, \textit{et al.} 2021. Zoonotic Risk Technology Enters the Viral Emergence Toolkit. \textit{Philosophical Transactions B} doi: 10.1098/rstb.2020.0360

\item Gibb,R, GF Albery, DJ Becker, L Brierley, R Connor, {\mefont TA Dallas}, EA Eskew, MJ Farrell, AL Rasmussen, SJ Ryan, A Sweeny, CJ Carlson, \& T Poisot. 2021. Data proliferation, reconciliation, and synthesis in viral ecology. \textit{BioScience}. doi: 10.1093/biosci/biab080

\item {\mefont Dallas, TA} \& P Jordano 2021. Species-area and network-area relationships in host-helminth interactions. \textit{Proceedings of the Royal Society B}. doi:10.1098/rspb.2020.3143

\item {\mefont Dallas, TA}, B Melbourne, G Legault, \& A Hastings. 2021. Initial abundance and stochasticity influence competitive outcome in communities. \textit{Journal of Animal Ecology} doi:10.1111/1365-2656.13485

\item Poisot,T, G Bergeron, K Cazelles, {\mefont TA Dallas}, D Gravel, A MacDonald, B Mercier, C Violet \& S Vissault. 2021. Global knowledge gaps in species interaction networks data. \textit{Journal of Biogeography} doi:10.1111/jbi.14127

\item {\mefont Dallas, TA}, M Saastamoinen, \& O Ovaskainen. 2021. Exploring the dimensions of metapopulation persistence: a comparison of structural and temporal measures. \textit{Theoretical Ecology} doi: 10.1007/s12080-020-00497-0 

\item {\mefont Dallas, TA} \& D Becker. 2021. Taxonomic resolution affects host-parasite association model performance. \textit{Parasitology} doi: 10.1017/S0031182020002371
\end{itemize}




{\yearfont 2020}

\begin{itemize}

\item {\mefont Dallas, TA}, L Santini, R Decker, \& A Hastings. 2020. Weighing the evidence for the abundant-centre hypothesis. \textit{Biodiversity Informatics}. doi: 10.17161/bi.v15i3.11989 

\item Carlson, CJ, Phillips, AJ, {\mefont TA Dallas}, Alexander, LW, Phelan, A, \& Bansal, S. 2020. What would it take to describe the global diversity of parasites?. \textit{Proceedings of the Royal Society B}. doi: 10.1098/rspb.2020.1841

\item {\mefont Dallas, TA}, B Melbourne, \& A Hastings. 2020. Community context and dispersal stochasticity drive variation in spatial spread. \textit{Journal of Animal Ecology}. doi: 10.1111/1365-2656.13331

\item {\mefont Dallas, TA}, L Holian, \& G Foster. 2020. What determines parasite species richness across host species? \textit{Journal of Animal Ecology}. doi: 10.1111/1365-2656.13276

\item {\mefont Dallas, TA} \& L Santini. 2020. The influence of stochasticity, landscape structure, and species traits on abundant-centre relationships. \textit{Ecography} doi:10.1111/ecog.05164

\item {\mefont Dallas, TA}, LH Antao, J Pöyry, R Leinonen, \& O Ovaskainen. 2020. Spatial synchrony is related to the rate of environmental change in Finnish moth communities. \textit{Proceedings of the Royal Society B}. doi: 10.1098/rspb.2020.0684

\item van Bergen, E, {\mefont TA Dallas}, DiLeo, MF, Kahilainen, A, Mattila, AL, Luoto, M, \& Saastamoinen, M. 2020. The effect of summer drought on the predictability of local extinctions in a butterfly metapopulation. Conservation Biology. doi: 10.1111/cobi.13515

\item \OA {\mefont Dallas, TA}, S Pironon, \& L Santini. 2020. Weak support for the abundant niche-centre hypothesis in North American birds. \textit{bioRxiv}. doi:10.1101/2020.02.27.968586

\item \OA Poisot, T, Bergeron, G, Cazelles, K, {\mefont TA Dallas}, Gravel, D, MacDonald, A, ... \& Vissault, S. 2020. Environmental biases in the study of ecological networks at the planetary scale. \textit{bioRxiv}. doi:10.1101/2020.01.27.921429

\end{itemize}





{\yearfont 2019}

\begin{itemize}

	\item {\mefont Dallas, TA},  M Saastamoinen, T Schulz, O Ovaskainen. 2019. The relative importance of local and regional processes to metapopulation dynamics. \textit{Journal of Animal Ecology}. doi: 10.1111/1365-2656.13141 

	\item \OA {\mefont Dallas, TA}, CJ Carlson, T Poisot. 2019. Testing predictability of disease outbreaks with a simple model of pathogen biogeography. \textit{Royal Society Open Science}. doi: 10.1098/rsos.190883

	\item {\mefont Dallas, TA}, Laine A-L, \& Ovaskainen O. 2019. Detecting parasite associations within multi-species host and parasite communities. \textit{Proceedings of the Royal Society B} \\ doi: 10.1098/rspb.2019.1109

	\item {\mefont Dallas, TA}, P{\"o}yry J, Leinonen R, Ovaskainen O. 2019. Temporal sampling and abundance measurement influences support for occupancy–abundance relationships. \textit{Journal of Biogeography} doi:10.1111/jbi.13718​

  \item Norberg, A, N Abrego Antia, F Guillaume Blanchet, FR Adler, BJ Anderson, J Anttila, MB Araújo, {\mefont TA Dallas}, D Dunson, J Elith, S Foster, R Fox, J Franklin, W Godsoe, A Guisan, B O'Hara, NA Hill, RD Holt, FKC Hui, M Husby, JA Kålås, A Lehikoinen, M Luoto, HK Mod, G Newell, I Renner, TV Roslin, J Soininen, W Thuiller, JP Vanhatalo, D Warton, M White, NE Zimmermann, D Gravel, and OT Ovaskainen. 2019. A comprehensive evaluation of predictive performance of 33 species distribution models at species \& community levels. \textit{Ecological Monographs} doi:10.1002/ecm.1370
    
  \item Cornelius Ruhs, E, Borden, DM, {\mefont TA Dallas}, \& E Pitman. 2019. Do feather traits convey information about bird condition during fall migration? \textit{Wilson Journal of Ornithology} doi:10.1676/18-174

  \item {\mefont Dallas, TA}, AL Gehman, AA Aguirre, SA Budischak, JM Drake, MJ Farrell, R Ghai, S Huang, \& I Morales-Castilla. 2019. Contrasting latitudinal gradients of body size in helminth parasites and their hosts. \textit{Global Ecology and Biogeography} doi: 10.1111/geb.12894

\item {\mefont Dallas, TA}, BA Han, CL Nunn, AW Park, PR Stephens, and JM Drake. 2018. Host traits associated with species roles in parasite sharing networks. \textit{Oikos} doi: 10.1111/oik.05602
 
\end{itemize}





{\yearfont 2018}
\begin{itemize}

\item {\mefont Dallas, TA}, BA Melbourne, \& A Hastings. 2018. When can competition and dispersal lead to checkerboard distributions? \textit{Journal of Animal Ecology} doi: 10.1111/1365-2656.12913

\item {\mefont Dallas, TA} \& A Hastings. 2018. Habitat suitability estimated by niche models is largely unrelated to species abundance. \textit{Global Ecology and Biogeography} doi: 10.1111/geb.12820

\item {\mefont Dallas, TA}, S Budischak, C Carlson, V Ezenwa, B Han, S Huang, AA Aguirre, \& PR Stephens. 2018. Gauging support for macroecological patterns in helminth parasites. \textit{Global Ecology and Biogeography} doi: 10.1111/geb.12819

\item {\mefont Dallas, TA}, R Decker, \& AM Hastings. 2018. Multiple data sources and freely available code is critical when investigating species distributions and diversity: a response to Knouft (2018). \textit{Ecology Letters} doi: 10.1111/ele.13105

\item {\mefont Dallas, TA}, A Gehman, \& MJ Farrell. 2018. Variable bibliographic database access could limit reproducibility. \textit{BioScience} doi:10.1093/biosci/biy074

\item Park, AW, MJ Farrell, JP Schmidt, S Huang, {\mefont TA Dallas}, P Pappalardo, JM Drake, PR Stephens, R Poulin, CL Nunn, \& TJ Davies. 2018. Characterizing the phylogenetic specialism-generalism spectrum of mammal parasites. \textit{Proceedings of the Royal Society B} doi: 10.1098/rspb.2017.2613

\item \OA {\mefont Dallas, TA}, JM Drake, \& M Krkosek. 2018. Experimental evidence of a pathogen invasion threshold. \textit{Royal Society Open Science} doi: 10.1098/rsos.171975

\item {\mefont Dallas, TA} \& T Poisot. 2018. Compositional turnover in host and parasite communities does not change network structure. \textit{Ecography} doi: 10.1111/ecog.03514

\end{itemize}




{\yearfont 2017}
\begin{itemize}

\item {\mefont Dallas, TA}, R Decker, \& AM Hastings. 2017. Species are not most abundant in the center of their geographic range or climatic niche. \textit{Ecology Letters} doi: 10.1111/ele.12860

\item Carlson, CJ, KR Burgio, {\mefont TA Dallas}, \& WM Getz. The Mathematics of Extinction Across Scales: From Populations to the Biosphere. In \textit{Mathematics of Planet Earth. Mathematics of Planet Earth}, vol 5. Springer. 

\item \OA Carlson,CJ, KR Burgio, ER Dougherty, AJ Phillips, VM Bueno, CF Clements, G Castaldo, {\mefont TA Dallas}, CA Cizauska, GS Cumming, J Do\~na, NC Harris, R Jovani, S Mironov, O Muellerklein, HC Proctor, \& WM Getz. 2017. Parasite biodiversity faces extinction and redistribution in a changing climate. \textit{Science Advances} doi: 10.1126/sciadv.1602422

\item {\mefont Dallas, TA}, S Huang, C Nunn, AW Park, \& JM Drake. 2017. Estimating parasite host range. \textit{Proceedings of the Royal Society B}. 284:1861. doi:10.1098/rspb.2017.1250.

\item \OA {\mefont Dallas, TA}, AW Park, \& JM Drake. 2017. Predicting cryptic links in host-parasite networks. \textit{PLoS Computational Biology}. 13(5): e1005557 doi:10.1371/journal.pcbi.1005557

\item \OA Evans, MV, {\mefont TA Dallas}, BA Han, CC Murdock, \& JM Drake. 2017. Data-driven identification of potential Zika virus vectors. \textit{eLife}. e22053. doi:10.7554/eLife.22053

\end{itemize}



{\yearfont 2016}
\begin{itemize}

\item \OA {\mefont Dallas, TA}, A Kramer, M Zokan, \& JM Drake. 2016. Ordination obscures the influence of environment on plankton metacommunity structure. \textit{Limnology and Oceanography Letters}. 54-61. doi:10.1002/lol2.10028

\item {\mefont Dallas, TA}, AW Park, \& JM Drake. 2016. Predictability of helminth parasite host range using information on geography, host traits and parasite community structure. \textit{Parasitology}. doi:10.1017/S0031182016001608

\item \OA {\mefont Dallas, TA} \& JM Drake. 2016. Fluctuating temperatures alter environmental pathogen transmission in a \textit{Daphnia}-pathogen system. \textit{Ecology and Evolution} 00: 1-8. doi:10.1002/ece3.2539

\item \OA Stephens, P, Altizer, S, Smith, K, Aguirre, A, Brown, J, Budischak, S, Byers, J, {\mefont TA Dallas}, Davies, J, Drake, J, Ezenwa, V, Farrell, M, Gittleman, J, Han, B, Huang, S, Hutchinson, R, Johnson, P, Nunn, C, Onstad, D, Park, A, Vazquez-Prokopec, G, Schmidt, J, \& R Poulin. 2016. The Macroecology of Infectious Diseases: A New Perspective on Global-scale Drivers of Pathogen Distributions and Impacts. \textit{Ecology Letters} 19(9): 1159-1171. doi: 10.1111/ele.12644

\item \OA {\mefont Dallas, TA}. 2016. \textit{helminthR}: An R interface to the London Natural History Museum's Host-Parasite Database. \textit{Ecography} 39(4): 391-393. doi: 10.1111/ecog.02131 

\item {\mefont Dallas, TA}, R Hall, \& J Drake. 2016. Competition-mediated feedbacks in experimental multi-species epizootics. \textit{Ecology} 97(3):661-670. doi:10.1890/15-0305.1 

\item \OA {\mefont Dallas, TA}, M Holtackers, \& J Drake. 2016. Costs of resistance and infection by a generalist pathogen. \textit{Ecology and Evolution} 6(6): 1737-1744. doi: 10.1002/ece3.1889 

\end{itemize}



{\yearfont 2015}

\begin{itemize}

\item \OA \ {\mefont Dallas, TA} \& E Cornelius. 2015. Co-extinction in a host-parasite network: identifying key hosts for network stability. \textit{Nature Scientific Reports} doi: 10.1038/srep13185

\item Park, AW, C Cleveland, {\mefont TA Dallas}, \& J Corn. 2015. Vector species richness increases hemorrhagic disease prevalence through functional diversity modulating the duration of seasonal transmission. \textit{Parasitology} 10: 1-6. doi: 10.1017/S0031182015000578

\item Presley SJ, {\mefont TA Dallas}, BT Klingbeil, \& MR Willig. 2015. Phylogenetic signals in host-parasite associations for Neotropical bats and Nearctic desert rodents. \textit{Biological Journal of the Linnean Society} 116(2): 312-327. 

\end{itemize}



{\yearfont 2014 and prior}

\begin{itemize}

\item \OA \ {\mefont Dallas, TA} \& JM Drake. 2014. Relative importance of environmental, geographic, and spatial variables on zooplankton metacommunities. \textit{Ecosphere} 5(9): art104 doi:10.1890/ES14-00071.1.

\item \OA \ {\mefont Dallas, TA}. 2014. \textit{metacom}: an R package for the analysis of metacommunity structure. \textit{Ecography} 37(4):402-405. doi:10.1111/j.1600-0587.2013.00695.x

\item {\mefont Dallas, TA} \& SJ Presley. 2014. Relative importance of host environment, transmission potential, and host phylogeny to the structure of parasite metacommunities. \textit{Oikos} 123: 866–874. doi:10.1111/oik.00707

\item \OA \ {\mefont Dallas, TA} \& JM Drake. 2014. Nitrate enrichment alters a Daphnia-microparasite interaction through multiple pathways. \textit{Ecology and Evolution} 4(3):243-250. doi: 10.1002/ece3.925

\item Kim, HJ, Cavanaugh, JE, {\mefont TA Dallas}, \& S For\'e. 2013. Model selection criteria for overdispersed data and their application to the characterization of a host-parasite relationship. \textit{Environmental and Ecological Statistics} doi:10.1007/s10651-013-0257-0

\item \OA \ {\mefont Dallas, TA}. 2013. \textit{metacom}: Analysis of the 'Elements of Metacommunity Structure'. R package version 1.2. http://CRAN.R-project.org/package=metacom

\item {\mefont Dallas, TA} \& S For\'e. 2013. Chemical attraction of \textit{Dermacentor variabilis} ticks parasitic to \textit{Peromyscus leucopus} based on host body mass and sex. \textit{Experimental and Applied Acarology} 61(2): 243-250. doi:10.1007/s10493-013-9690-x

\item {\mefont Dallas, TA}, S For\'e, \& HJ Kim. 2012. Modeling the influence of \textit{Peromyscus leucopus} body mass, sex and habitat on immature \textit{Dermacentor variabilis} burdens. \textit{Journal of Vector Ecology}. 37(2):338-341.doi:10.1111/j.1948-7134.2012.00236.x

\item {\mefont Dallas, TA}, S For\'e, \& HJ Kim. 2010. Factors influencing immature \textit{Dermacentor variabilis} load on the white-footed mouse (\textit{Peromyscus leucopus}). \textit{Technical Report, Truman State University}.
\end{itemize}










\section{\faCode \ \  software}
\begin{entrylist}
 \entry
 {\href{http://cran.r-project.org/web/packages/metacom/}{\textbf{metacom}}}
 {Analysis of metacommunity structure} 
 {R package (author)}

 \entry
 {\href{https://github.com/viralemergence/insectDisease}{\textbf{insectDisease}} \ \ }
 {Access to the Ecological Database of the World's Insect Pathogens}
 {R package (author)}

 \entry
 {\href{https://cran.r-project.org/web/packages/helminthR/index.html}{\textbf{helminthR}} \ \ }
 {Portal to London Natural History Museum host-helminth database}
 {R package (author)}

 \entry
 {\href{https://cran.r-project.org/web/packages/Hmsc/index.html}{\textbf{Hmsc}} \ \ }
 {Hierarchical modeling of species communities}
 {R package (author)}

 \entry
 {\href{http://github.com/cjcarlson/spatExtinct}{\textbf{spatExtinct}} \ \ }
 {Spatially interpolated extinction date estimation}
 {R package (contributor)}

\end{entrylist}





\section{\faVideoCamera \ \ \ presentations}

\begin{itemize}

\item {\mefont T Dallas}. \textit{Invited seminar to University of South Carolina's "Mathematical Foundations of Data Science" group}. October 2022. 

\item {\mefont T Dallas}. \textit{Departmental seminar at University of South Carolina}. October 2022. 

\item {\mefont T Dallas}. \textit{Invited seminar at Duke University}. Hosted by Jean-Philipe Gibert. September 2022. 
 
\item {\mefont T Dallas} Ecological Society of America Meeting. August 2022.

\item {\mefont T Dallas}, C Ten Caten, L Holian. British Ecological Society Macroecology meeting. July 2022.

\item {\mefont T Dallas}, G Foster, R Richards, and B Elderd. Ecology and Evolution of Infectious Disease meeting. June 2022.  

\item {\mefont T Dallas} and B Elderd. \textit{Invited talk at} "Science and Spirits" at LSU. November 2021. 

\item {\mefont T Dallas}. \textit{Invited seminar at University of South Carolina}. Hosted by Tammi Richardson. May 2021. 

\item {\mefont T Dallas}. \textit{Invited seminar at Truman State University}. Student invited speaker. April 2021. 

\item {\mefont T Dallas}. \textit{Invited seminar at International University of Rijeka}. Hosted by Danijel Krismanic. June 2019.

\item {\mefont T Dallas}. \textit{Invited seminar at Osnabr\"uck University}. Hosted by Frank Hilker. December 2018.

\item {\mefont T Dallas}. \textit{Invited seminar at McGill University}. Hosted by Rowan Barrett. April 2018. 

\item {\mefont T Dallas}. \textit{Invited seminar at University of Arkansas}. Hosted by John David Wilson. February 2018. 

\item {\mefont T Dallas}. \textit{Invited seminar at Louisiana State University}. Hosted by Bret Elderd. January 2018. 

\item {\mefont T Dallas}. \textit{Invited seminar at University of California - Los Angeles}. Hosted by Jamie Lloyd-Smith. January 2018. 

\item {\mefont T Dallas}, B Melbourne, G Legault, A Hastings. Initial abundance and stochasticity influence species coexistence \textit{Society for Mathematical Biology}, July 2017.

\item {\mefont T Dallas} and JM Drake. Using niche modeling to detect unobserved interactions in host-parasite networks. \textit{Ecological Society of America}, August 2015.

\item JE Byers, P Pappalardo, JP Schmidt, PR Stephens, S Haas, C Nunn, JM Drake, and {\mefont T Dallas}. What parasite and host traits best explain the geographic range of mammal parasites and diseases? \textit{Ecological Society of America}, August  2015.

\item {\mefont T Dallas} and JM Drake. Costs of resistance and infection in \textit{Daphnia} species exposed to a generalist microparasite. \textit{Ecology and Evolution of Infectious Disease Conference}. Fort Collins, CO. June 2014

\item  {\mefont T Dallas}, JM Drake, M Krkosek. Thresholds to pathogen invasion: theory + experiment. \textit{Ecological Society of America}. Sacramento, California. August 2014

\item {\mefont T Dallas} and JM Drake. The Influence of Nitrate on Fungal Parasitism of \textit{Daphnia}. \textit{98th annual American Society for Microbiology (Southeastern Branch)}. October 2012.

\item {\mefont T Dallas}. Effects of competition and selective predation in a two-host system. \textit{Odum School of Ecology Graduate Student Symposium}. Athens GA. January 2011.

\item {\mefont T Dallas}. Thesis defense: An examination of variation in \textit{Dermacentor variabilis} burdens within and between host species. \textit{Truman State University}. August 2010.

\end{itemize}










\section{\faVideoCamera \ \ \ meeting participation}

\begin{entrylist}

  \entry
    {2022-}
		{British Ecological Society Macroecology Special Interest Group}
    {Co-organized 10 year anniversary plenary and was invited speaker on predictive macroecology}

  \entry
    {2020-}
		{Ecological Forecasting Initiative }
    {Co-designer of beetle forecast challenge}

  \entry
    {2021}
		{Ecological Forecasting Initiative}
    {Empowering Development of the Next Generation of Educational Materials for Ecological Forecasting}

  \entry
    {2021}
		{BES Macroecology meeting}
    {Panel participant on early career transitions}


\iffalse
  \entry
    {2022}
		{}
    {} 
\fi

\end{entrylist}












\section{ \faInstitution \ \ \ teaching}

\begin{entrylist}

  \entry
    {spring 2023}
		{Population Biology (Biol 763/Math XXX)} % or {Theoretical Ecology (Biol 765)}
    {University of South Carolina}


  \entry
    {spring 2022}
		{Reproducible Research in R (Biol 599)}
    {University of South Carolina}

  \entry
    {2020}
		{Vector-borne disease (Biol 7901)}
    {Louisiana State University}

  \entry
    {2020}
		{Reproducible Research in R (Biol 4800)}
    {Louisiana State University}

  \entry
    {2019, 2021}
		{Principles of Ecology (Biol 4253)}
    {Louisiana State University}

\end{entrylist}










\section{\faDollar \ \ grants}

\begin{entrylist}

% \entry
% {2022-2024}
% {Portugal grant}
% {Portugal; \$400000 (collaborator)}

% \entry
% {2022-202x}
% {U of South Carolina}
% {Aspire track 1; \$15000 (PI)}

% \entry
% {2022-2023}
% {U of South Carolina}
% {Big Data Health Science Pilot grant; \$31000 (PI)}

% \entry
% {2022-2025}
% {National Science Foundation - DEB}
% {The effects of species traits and geography on temporal variation in population dynamics; \$487000}

% \entry
% {2023-2026}
% {National Science Foundation - CYBER}
% {Towards a wormier world: Augmenting and georeferencing the largest host-helminth database; \$622,413}


% \entry
% {2022-2025}
% {National Science Foundation - CAREER}
% {  \$ }


% \entry
% {2022-202x}
% {NSF}
% {NSF; \$400000 (PI)}

 \entry
 {2021 - }
 {Actively engaging students in hardware and software development}
 {LSU Foundation and LSU College of Science; \$44,000 (PI)}

 \entry
 {2020-2022}
 {RAPID: Epidemic control strategies for COVID-19 in age-structured populations: A multi-model approach }
 {NSF RAPID; \$200,000 (PI)}

 \entry
 {2020-2022}
 {BII-Design: Exploring the ecology and evolution of the global virome with big data and machine learning}
 {NSF Bio Institute - Design; \$166,189 (co-PI)}

 \entry
 {2020-2023}
 {MSA: Understanding spatial patterns of abundance and occupancy in terms of taxa, traits, and space}
 {NSF Macrosystems and NEON Science; \$274,542 (PI)}

% \entry
% {2020-2021}
% {Modeling coronavirus ecology for pandemic preparedness}
% {IVADO; \$33,000 (collaborator)}

\end{entrylist}






%%%% -----------
%\section{\faDollar \ \ nonfunded grants}

% \entry
% {2021}
% {Predicting disease emergence and elimination thresholds in complex networks}
% {Exploratory Data Science Methods and Algorithm Development in Infectious and Immune-mediated Diseases (R21) (RFA-AI-21-035); \$212000 }


% \entry
% {2020}
% {sNetwork: the macroecology of species interaction networks}
% {sDiv working group; \$ }

% \entry
% {2019}
% {US-UK Collab: DEMETeR: Dynamic Evolutionary Models of Ecological Transmission informed by eDNA in Rivers to understand aquatic disease outbreaks}
% {National Science Foundation EEID; \$267,703}

% \entry
% {2019}
% {Understanding large-scale patterns of helminth parasite diversity}
% {LA Board of Regents (BOR-RCS); \$258,079}


%\end{entrylist}





\section{ \faUserPlus \ \ \ professional service}

\iffalse
\textbf{Internal service and grant review panels}

\begin{entrylist}

% info for internal but don't really need to include it in the main CV. 
  \entry{October 2022}
  {Host of seminar speaker}
  {Lauren Shoemaker (U Wyoming)}

  \entry{2022}
  {Reference letters}
  {see refLetters.csv}

  \entry{January 2022}
  {Letter of collaboration}
  {Katie Kathrein for NIH grant}

  \entry{February 2022}
  {NSF DEB}
  {NSF \textit{ad hoc} grant reviewer}

  \entry{May 2022}
  {NSF URoL: Emergent Networks}
  {NSF grant panel reviewer}

  \entry{2022}
  {U of SC undergraduate research symposium}
  {judge}

  \entry{2022}
  {U of SC Faculty Senate}
  {member/rep}

  \entry
	{2020}
	{Undergraduate Biol Welcome Week planning committee}
	{Evanna Gleason, David Vinyard, Karen Maruska}

\end{entrylist}

\fi









\textbf{Editorial positions and professional affiliations}

\begin{entrylist}
 
 \entry
	 {2020-}
	 {Ecology Letters}
	 {Editor}
	 
 \entry
	 {2019-}
	 {Ecosphere}
	 {Editor - disease track}

 \entry
	 {2019-}
	 {Github Education}
	 {Campus advisor}

 \entry
	 {2019-}
	 {The Carpentries}
	 {Instructor}

 \entry
	 {2019-}
	 {LIFEPLAN: A Planetary Inventory of Life}
	 {Sampling site}

\end{entrylist}



For information on my service, see my {\href{https://publons.com/author/904038/tad-dallas#profile}{Publons page}. I have served as a reviewer for the following journals:

\begin{multicols}{2}
\begin{itemize}
 \item African Journal of Wildlife Research
 \item American Naturalist
 \item Applied Network Science
 \item Basic and Applied Ecology
 \item Biological Conservation
 \item Ecography
 \item Ecology
 \item Ecology and Evolution
 \item Ecology Letters
 \item Ecological Complexity
 \item EcoHealth
 \item Ecosphere
 \item Functional Ecology
 \item Freshwater Biology
 \item Global Change Biology
 \item Global Ecology and Biogeography
 \item International Journal of Parasitology
 \item Invertebrate Biology
 \item Journal of Animal Ecology
 \item Journal of Biogeography
 \item Journal of Ecology
 \item Journal of Natural History
 \item Journal of Vector Ecology
 \item Landscape Ecology
 \item Methods in Ecology and Evolution
 \item Nature Ecology \& Evolution
 \item Oecologia
 \item Oikos
 \item Parasitology
 \item Philosophical Transactions B
 \item PLoS One
 \item Proceedings of the Royal Society B
 \item Proceedings of the National Academy of Sciences
 \item Scientometrics
 \item Theoretical Ecology
\end{itemize}
\end{multicols}



%\begin{itemize}
% \item  {\href{http://esa.org/disease}{Ecological Society of America - Disease Ecology section}}
% \item  {\href{http://diseasemacroecology.ecology.uga.edu/}{Macroecology of Infectious Disease - NSF Research Coordination Network}}
% \item  {\href{http://daphnia.ecology.uga.edu/ceesg}{Computational Ecology and Epidemiology Study Group - UGA}}
% \item  {\href{http://gsa.ecology.uga.edu}{Graduate Student Association - Odum School of Ecology}}
%\end{itemize}





\section{\faTrophy \ \ awards}
\begin{entrylist}

 \entry{2022}
 {Echo 25 award; Truman State University}
 {}
 

 \entry
 {2021}
 {LSU Non-Tenured Faculty Research Award}
 {\$1000}

\end{entrylist}











\section{ \faUserPlus \ \ \ community engagement}

\begin{entrylist}

%  \entry{}
%  {}
%  {}

%  \entry{}
%  {}
%  {}

%  \entry{}
%  {}
%  {}

 \entry
	 {2022}
	 {Virtual Kitchen Lab}
	 {interview/podcast}

 \entry
	 {2021}
	 {LSU Science and Spirits podcast}
	 {interview/podcast}
 
 \entry
	 {2021-}
	 {Futures Fund coding instructor}
	 {https://www.thewallsproject.org/futuresfund}

 \entry
	 {2019-2022}
	 {Front Yard Bikes volunteer}
	 {https://www.frontyardbikes.com/}

\end{entrylist}















\section{\faUsers \ \ mentoring}
\begin{entrylist}

 \entry
 {2022 - }
 {Masters thesis committee, U of SC, Arnold School of Public Health}
 {Kayla Bramlett}
 
 \entry
 {2022 - }
 {Doctoral dissertation committee, U of SC}
 {Alexander Barth}

 \entry
 {2021 - }
 {Doctoral dissertation committee, U of SC, SEOE}
 {Birch Lazo-Murphy}

 \entry
 {2021 - }
 {Doctoral dissertation committee, LSU}
 {Wissam Jawad}

 \entry
 {2021 - }
 {Dissertation committee chair, U of SC}
 {Lauren Holian}

 \entry
 {2020 - }
 {Dissertation committee chair, U of SC}
 {Grant Foster}

 \entry
 {2019 - }
 {Dissertation committee chair, U of SC}
 {Cleber Ten Caten}
 
 \entry
 {2019 - }
 {Doctoral dissertation committee, LSU}
 {Jason Janeaux}

 \entry
 {2014}
 {Population Biology of Infectious Disease REU}
 {Trianna Humphries}

 \entry
 {2013}
 {Young Dawgs Program}
 {Mathieu Holtackers}

\end{entrylist}












\iffalse
\section{\faGlobe \ \ professional affiliations}
\begin{entrylist}

	\entry
	{2017 - 2020 \ \ \ \ \ \ \ \ }
  {Society for Mathematical Biology}
  {}

	\entry
  {2016 - 2018  \ \ \ \ \ \ \ }
  {Association for the Sciences of Limnology and Oceanography}
  {}

	\entry
  {2012 - }
  {Ecological Society of America member}
  {Aquatic Ecology and Disease Ecology sections}

\end{entrylist}
\fi


\end{document}
